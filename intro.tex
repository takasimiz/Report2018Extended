\section{Introduction}

\subsection{Objective}
Our goal is to classify existing optimization methods by a new equivalence relation using the Effectus logic  \cite{Jacobs2015NewLogic}. The relation which induces equivalence classes, namely a classification of the algorithms, captures every aspect of optimization: single/multi-modality; robustness; single/multi-objectiveness; noisy observation; posterior expected loss; and mixtures of combinatorial and continuous design variables. 

This study leads to a construction of a foundation of ``the optimization theory'' which has not yet existed until today. Although, there is one for a unimodal case, so-called ``mathematical optimization theory,'' which is a special case of the ground theory.

\subsection{Quantum cognition and Effectus logic}
Over the past decade, engineers and researchers have been widely utilized Bayesian method to combat ill-conditioned problems in engineering.
They claim a Bayesian model can adequately simulate various human behaviors, such as decision making, similarity judgment, and conceptional combination, even though there is no substantial evidence to back the assumption and psychological experiments indicate otherwise, see e.g. \cite{Tversky1983ExtensionalJudgment}. 

Recently, psychologists found a model with the Born rule -- one of the cornerstones of quantum mechanics -- fits human behaviors better than the ones with the Kolmogorov's axiom including Bayesian probability, see e.g. \cite{Busemeyer2011AErrors, Pothos2013CanModeling, Denolf2017AGames}.

In order to apply the novel finding in psychology into engineering, so that one can minimize an expected loss for a posterior distribution in which some human behaviors are involved, we use the Effectus theory \cite{Jacobs2015NewLogic, Jacobs2017QuantumCognition}, which can manage Bayesian and quantum probabilities in a unified fashion.


