\title{An optimal control of irrational agents}

\author{Satoru Takasimizu}

%\documentclass[final, twocolumn, 15pt]{article}

\documentclass[final, 15pt]{article}
%\documentclass[draft, a4paper, 17pt]{extarticle}

%\documentclass[a4paper, 20pt]{extarticle}
%\documentclass[a4paper, twocolumn, 20pt]{extarticle}
%\documentclass[a4paper, twocolumn, 17pt]{extarticle}
%\documentclass[a4paper, twocolumn, 15pt]{extarticle}
%\documentclass[a4paper, 16pt]{extarticle}

%\usepackage[utf8]{inputenc}
%\usepackage[english]{babel}

%\usepackage[
%backend=biber,
%style=alphabetic,
%sorting=ynt
%]{biblatex}
%\usepackage[
%backend=biber,
%style=alphabetic,
%citestyle=authoryear
%]{biblatex}
\usepackage[
backend=biber,
style=alphabetic,
sorting=ynt
]{biblatex}
%\bibliographystyle{acm}
%\usepackage{biblatex}
%\addbibresource{references.bib}
\addbibresource{refs.bib}
\usepackage{ifdraft}

\usepackage[prependcaption, colorinlistoftodos, textsize=small, textwidth=4.0cm, obeyDraft]{todonotes}
\setlength{\marginparwidth}{4.0cm}
%\usepackage[tiny]{titlesec}
%\usepackage[small]{titlesec}
%\usepackage[medium]{titlesec}
\usepackage{xifthen}

%\usepackage{graphicx}
\usepackage{tikz-cd, mathrsfs}

\usepackage[nohead, nofoot]{geometry}
\usepackage{xparse, soul, color, amsmath, amsthm, amssymb, physics, stmaryrd}

\usetikzlibrary{arrows}
\usepackage{tikz}

\numberwithin{equation}{section}
\usepackage{mathtools}
%\usepackage{xargs}

\usepackage{hyperref}
\hypersetup{
    colorlinks=true,
    linkcolor=blue,
    filecolor=magenta,      
    urlcolor=cyan,
}
\ifdraft{
\geometry{
% paperheight=330mm 
 a4paper,
 left=10mm,
% right=80mm,
 right=50mm,
 top=10mm,
 bottom=20mm
}}{
\geometry{
% paperheight=330mm 
 a4paper,
 left=20mm,
% right=80mm,
 right=20mm,
 top=20mm,
 bottom=20mm
}}

\usepackage{ebproof}

\newcommand{\pr}[1]{\mathrm{Pr}(#1)}

\newcommand{\Cs}{\Bbb{C}}
\newcommand{\Rs}{\Bbb{R}}

\newcommand{\Kl}[1]{\mathrlap{\mathcal{K}} \, \, \, \, \ell (#1)}
\newcommand{\DM}[1]{\mathrlap{\mathcal{D}} \, \, \, \mathcal{M} (#1)}

\newcommand{\ctSets}{\mathbf{Sets}}
\newcommand{\ctMon}{\mathbf{Mon}}
\newcommand{\ctBA}{\mathbf{BA}}
\newcommand{\ctGrp}{\mathbf{Grp}}

\newcommand{\ctC}{\mathscr{C}}
\newcommand{\ctD}{\mathscr{D}}

%\newcommand{\textunderscript}[1]{$_{\text{#1}}$}
%\newcommand{\ctCstar}[1]{$\mathbf{Cstar}_{\text{#1}}$}
%\newcommand{\ctCstar}{\mathbf{Cstar}}

\NewDocumentCommand{\ctCstar}{m}{
\IfValueTF{#1}
{\ensuremath{\mathbf{Cstar}{}_\text{#1}}}%
{\ensuremath{\mathbf{Cstar}}}}

\NewDocumentCommand{\ctFH}{m}{
\IfValueTF{#1}
{\ensuremath{\mathbf{FdHilb}{}_\text{#1}}}%
{\ensuremath{\mathbf{FdHilb}}}}

\newcommand\setOf[1]{\ensuremath{\Bbb{#1}}}

\NewDocumentCommand{\Ct}{m o}
{\ensuremath{\mathbf{#1}\IfValueT{#2}{(#2)}}}

\NewDocumentCommand{\ct}{m o}
{\mathscr{#1}\IfValueT{#2}{(#2)}}

\NewDocumentCommand{\fr}{m o}
{\mathcal{#1}\IfValueT{#2}{(#2)}}

%\newcommand{\opposite}[1]{\mathbf{#1}^\mathrm{op}}
\newcommand{\opposite}[1]{\ensuremath{\mathbf{#1}^\mathrm{op}}}

%\usepackage[footnotes,definitionLists,hashEnumerators,smartEllipses,hybrid]{markdown}
\usepackage[fencedCode,inlineFootnotes,citations,definitionLists,hashEnumerators,smartEllipses,hybrid]{markdown}

% https://tex.stackexchange.com/questions/126711/
%\newcommand{\sslash}{\mathbin{/\mkern-6mu/}}

\newtheorem{Th}{Theorem}[section]
\newtheorem{Lem}{Lemma}[section]
\newtheorem{Claim}{Claim}[section]
\theoremstyle{definition}
\newtheorem{Def}{Definition}[section]
\newtheorem{Eg}{Example}[section]
\theoremstyle{remark}
\newtheorem{rem}{Remark}[section]

\newenvironment{di*}[1][]
{\[\begin{tikzcd}[#1]}
{\end{tikzcd}\]}

\newenvironment{di}[2][]
{\begin{equation}\begin{tikzcd}[#1]\label{di:#2}}
{\end{tikzcd}\end{equation}}
%\NewDocumentEnvironment{di}{O{} o}
%{\begin{equation}\begin{tikzcd}[#1]\IfNoValueT{#2}{\label{di:#2}}}
%{\end{tikzcd}\end{equation}}

\newcommand{\hlcyan}[1]{{\sethlcolor{cyan}\hl{#1}}}


\newcommand{\todoEng}[1]{%
\ifthenelse{\isempty{#1}}{%
{\color{red}{(FILL)}}\addcontentsline{tdo}{todo}{Fill there}%
}{%
{\copplor{red}{(FILL: #1)}}\addcontentsline{tdo}{todo}{Fill there: #1}%
}}

%\newcommand{\todoFix}[2][fancyline, color=green]{\todoLinked[#2]{Fix}{#2}}
%\newcommand{\todoFilll}[2][fancyline, color=orange]{\todoLinked[#1]{CAP}{#2}}

\NewDocumentCommand{\Fix}{O{} m}{%
\todoLinked[noprepend, color=yellow, #1]{Fix: #2}{#2}}

%\NewDocumentCommand{\todoFill}{o}{%
%\ifdraft%
%{  \IfValueTF{#1}{%
%  {\sethlcolor{green}\hl{#1}} %
%  \todo[color=green]{Fill: #1}%
%  }{%
%  {\sethlcolor{green}\hl{(FILL)}} %
%  \todo[color=green]{Fill}%
%  }%
%}{}}

\NewDocumentCommand{\todoFill}{O{} m}{%
\ifdraft{
\ifthenelse{\isempty{#2}}{%
{\sethlcolor{green}\hl{(FILL)}}%
\todoLinked[noprepend, color=green, #1]{Fill: fill}{Fill}%
}{%
{\sethlcolor{green}\hl{#2}}%
\todoLinked[noprepend, color=green, #1]{Fill: #2}{#2}%
}}{}}


\NewDocumentCommand{\memo}{O{} O{} m}{%
\todoLinked[inline, noprepend, color=yellow, #1]{Memo: #2}{%
\begin{minipage}{0.9\linewidth}#3\end{minipage}}}

\NewDocumentCommand{\todoTemp}{m m m o o}{%
    \ifthenelse{\isempty{#4}}{{\setulcolor{#3}\ul{#1}}}{}%
    \todoLinked[color=#3, fancyline]{#2}{#5}}

%\NewDocumentCommand{\todoT}{o o}{%
%    \todoTemp{(FILL)}{Fill}{red}[#1][#2]}

\usepackage{multicol}

\newcommand{\tcd}[2][]{
\begin{tikzcd}[ampersand replacement=\&, #1]
#2
\end{tikzcd}}

\usepackage{tabularx}

\tikzset{
  no line/.style={draw=none,
    commutative diagrams/every label/.append style={/tikz/auto=false}}}

% https://tex.stackexchange.com/questions/187444
\newcommand{\nospacepunct}[1]{\makebox[0pt][l]{\,#1}}

%\usepackage{luacode}

%\newcommand{\adj}[1]{%
%\rotatebox[origin=c]{\luadirect{tex.print(180 + %#1)}}{$\vdash$}}


\NewDocumentCommand{\equ}{s O{} m O{}}{%
\IfBooleanTF{#1}{\[}
{\begin{equation}\ifthenelse{\isempty{#2}}{}{\label{eq:#2}}}
#3\nospacepunct{#4}
\IfBooleanTF{#1}{\]}{\end{equation}}}


\NewDocumentCommand{\tce}{s O{} O{} m O{}}{%
\IfBooleanTF{#1}{\[}
{\begin{equation}\ifthenelse{\isempty{#3}}{}{\label{di:#3}}}%
\begin{tikzcd}[ampersand replacement=\&, #2]#4\end{tikzcd}
#5%
\IfBooleanTF{#1}{\]}{\end{equation}}}


\NewDocumentCommand{\dieq}{s O{} m O{:=} m}{%
\IfBooleanTF{#1}{\[}
{\begin{equation}\ifthenelse{\isempty{#2}}{}{\label{di:#2}}}
\begin{array}{ccc}
#3&#4&#5
\end{array}
\IfBooleanTF{#1}{\]}{\end{equation}}
}


\newcounter{todoListItems}
\newcommand{\todoLinked}[3][]{
\addtocounter{todoListItems}{1}
\todo[caption={\protect\hypertarget{todo\thetodoListItems}{}#2}, #1]
{#3\hfill \hyperlink{todo\thetodoListItems}{$\uparrow$}}}


\begin{document}%%%%%%%%%%%%%%%%%%%%%%%%%%%%%%%%
\maketitle
\tableofcontents\label{toc}

%
\section{Sandbox}

\begin{tikzcd}
\cdot \ar[Rightarrow]{r}{f} \ar{rd}{a}  & \cdot \ar{d}{g} \ar{rd}{r}\\
                            & \cdot \ar{r}{h}   & \cdot
\end{tikzcd}

\begin{tikzcd}
A & B \\
  & C & D
\ar[from=1-1, to=1-2]{}{f}
\ar[from=1-2, to=2-3]{}{h \circ g}
\ar[from=1-1, to=2-2]{}[swap]{g \circ f}
\ar[from=2-2, to=2-3]{}[swap]{h}
\ar[from=1-2, to=2-2]{}{g}
\end{tikzcd}

\begin{tikzcd}
F & G \\
  & F & G
\ar[Rightarrow, from=1-1, to=1-2]{}{\tau}
\ar[Rightarrow, from=1-2, to=2-3]{}{1_{\ct{C}}}
\ar[Rightarrow, from=1-1, to=2-2]{}[swap]{1_{\ct{D}}}
\ar[Rightarrow, from=2-2, to=2-3]{}[swap]{\sigma}
\ar[Rightarrow, from=1-2, to=2-2]{}{\theta}
\end{tikzcd}

\begin{tikzcd}
\ct{C}  & \ct{D} \\
        & \ct{C} & \ct{D}
\ar[from=1-1, to=1-2]{}{F}
\ar[from=1-2, to=2-3]{}{1_{\ct{D}}}
\ar[from=1-1, to=2-2]{}[swap]{1_{\ct{C}}}
\ar[from=2-2, to=2-3]{}[swap]{F}
\ar[from=1-2, to=2-2]{}{G}
\end{tikzcd}

\begin{tikzcd}
\ctC \ar[bend left=20]{r}[name=F]{F}
   & \ctD \ar[bend left=20]{l}[name=U]{U}
               \ar[from=F, to=U, no line]{}{\adj{270}}
\end{tikzcd}

\noindent
\begin{equation}
\begin{tabularx}{\textwidth}{%
|>{\centering\arraybackslash}m{0.3\textwidth}%
|>{\centering\arraybackslash}m{0.05\textwidth}%
|>{\centering\arraybackslash}m{0.3\textwidth}|}
\tcd{TA \ar{d}{f} \& B\\ A} & is & \tcd{TB}
\end{tabularx}
\end{equation}
\markdownInput{sb.md}
%\section{Introduction}

\subsection{Objective}
Our goal is to classify existing optimization methods by a new equivalence relation using the Effectus logic  \cite{Jacobs2015NewLogic}. The relation which induces equivalence classes, namely a classification of the algorithms, captures every aspect of optimization: single/multi-modality; robustness; single/multi-objectiveness; noisy observation; posterior expected loss; and mixtures of combinatorial and continuous design variables. 

This study leads to a construction of a foundation of ``the optimization theory'' which has not yet existed until today. Although, there is one for a unimodal case, so-called ``mathematical optimization theory,'' which is a special case of the ground theory.

\subsection{Quantum cognition and Effectus logic}
Over the past decade, engineers and researchers have been widely utilized Bayesian method to combat ill-conditioned problems in engineering.
They claim a Bayesian model can adequately simulate various human behaviors, such as decision making, similarity judgment, and conceptional combination, even though there is no substantial evidence to back the assumption and psychological experiments indicate otherwise, see e.g. \cite{Tversky1983ExtensionalJudgment}. 

Recently, psychologists found a model with the Born rule -- one of the cornerstones of quantum mechanics -- fits human behaviors better than the ones with the Kolmogorov's axiom including Bayesian probability, see e.g. \cite{Busemeyer2011AErrors, Pothos2013CanModeling, Denolf2017AGames}.

In order to apply the novel finding in psychology into engineering, so that one can minimize an expected loss for a posterior distribution in which some human behaviors are involved, we use the Effectus theory \cite{Jacobs2015NewLogic, Jacobs2017QuantumCognition}, which can manage Bayesian and quantum probabilities in a unified fashion.



\listoftodos[(Todos)]
%Quantum prob 101 for bayes reserachers

\hypertarget{the-born-rule}{%
\section{The Born rule}\label{the-born-rule}}

\hypertarget{the-born-rule-for-finite-dimensional-case-is}{%
\subsection{The Born rule for finite-dimensional case
is}\label{the-born-rule-for-finite-dimensional-case-is}}

\[Pr(p|\rho) = Tr(\rho p)\] @label-eq:born or
\[Pr(p|\rho) = Tr(\rho p)\] @label-eq:born
\[(\rho \vDash p) = Tr(\rho p)\] @label-eq:born Recap: classical prob
(Bayes):

\hypertarget{recall-the-bayes-theorem-for-discrete-distribution-omega-is}{%
\subsection{\texorpdfstring{Recall the Bayes' theorem for discrete
distribution \[\omega\]
is}{Recall the Bayes' theorem for discrete distribution \textbackslash{}omega is}}\label{recall-the-bayes-theorem-for-discrete-distribution-omega-is}}

\hypertarget{prqp-prpqprqprp-label-eqbayes}{%
\subsubsection{\texorpdfstring{\[Pr(q|p) = Pr(p|q)Pr(q)/Pr(p)\]
@label-eq:bayes}{Pr(q\textbar{}p) = Pr(p\textbar{}q)Pr(q)/Pr(p) @label-eq:bayes}}\label{prqp-prpqprqprp-label-eqbayes}}

where \[p, q \sim \omega\] and \[Pr(q) \neq 0\].

\hypertarget{this-equality-is-based-on-the-commutatibity-of-classical-prob.}{%
\subsubsection{This equality is based on the commutatibity of classical
prob.}\label{this-equality-is-based-on-the-commutatibity-of-classical-prob.}}

\hypertarget{prqpprq-prp-q-prq-p-prpqprp.}{%
\paragraph{\texorpdfstring{\[Pr(q|p)/Pr(q) = Pr(p \& q) = Pr(q \& p) = Pr(p|q)/Pr(p)\].}{Pr(q\textbar{}p)/Pr(q) = Pr(p \textbackslash{}\& q) = Pr(q \textbackslash{}\& p) = Pr(p\textbar{}q)/Pr(p).}}\label{prqpprq-prp-q-prq-p-prpqprp.}}

\hypertarget{where}{%
\subparagraph{where}\label{where}}

\[p \& q = \sqrt{p}q\sqrt{p}\]. In order to represent @label-eq:bayes,
\#? In the context of quantum probability, bayes theorem'
@label-eq:bayes is

\hypertarget{let-a-b-ab---ba}{%
\subsection{\texorpdfstring{Let
\[[A, B] := AB - BA\]}{Let {[}A, B{]} := AB - BA}}\label{let-a-b-ab---ba}}

Theorem \[[A,B] =0\] iff (\#{[}Jac15{]}, Th??)

\hypertarget{if-pq-0-i.e.-p-and-q-are-commutative}{%
\subsubsection{\texorpdfstring{if \[[p,q] = 0\], i.e. \[p\] and \[q\]
are
commutative,}{if {[}p,q{]} = 0, i.e. p and q are commutative,}}\label{if-pq-0-i.e.-p-and-q-are-commutative}}

\hypertarget{omega-vdash-p-q-tromega-sqrt-p-q-sqrt-p-tromega-sqrt-p-sqrt-p-q-tromega-p-q-tromega-p-sqrt-q-sqrt-q-tromega-sqrt-q-p-sqrt-q-omega-vdash-q-p}{%
\paragraph{\texorpdfstring{\[(\omega \vDash p \& q) = Tr(\omega \sqrt p q \sqrt p) = Tr(\omega \sqrt p \sqrt p q) = Tr(\omega p q) = Tr(\omega p \sqrt q \sqrt q) = Tr(\omega  \sqrt q p \sqrt q) = (\omega \vDash q \& p)\]}{(\textbackslash{}omega \textbackslash{}vDash p \textbackslash{}\& q) = Tr(\textbackslash{}omega \textbackslash{}sqrt p q \textbackslash{}sqrt p) = Tr(\textbackslash{}omega \textbackslash{}sqrt p \textbackslash{}sqrt p q) = Tr(\textbackslash{}omega p q) = Tr(\textbackslash{}omega p \textbackslash{}sqrt q \textbackslash{}sqrt q) = Tr(\textbackslash{}omega  \textbackslash{}sqrt q p \textbackslash{}sqrt q) = (\textbackslash{}omega \textbackslash{}vDash q \textbackslash{}\& p)}}\label{omega-vdash-p-q-tromega-sqrt-p-q-sqrt-p-tromega-sqrt-p-sqrt-p-q-tromega-p-q-tromega-p-sqrt-q-sqrt-q-tromega-sqrt-q-p-sqrt-q-omega-vdash-q-p}}

then @label-eq:bayes hold.

\hypertarget{let}{%
\paragraph{Let}\label{let}}

\[s = diag((s_i)_i)\] \[p = diag((p_i)_i)\] \[q = diag((q_i)_i)\]
\[\omega|_p \vDash q\] is

\hypertarget{omega_p-vdash-q-omega-vdash-p-q-omega-vdash-p-tromega-sqrt-p-q-sqrt-ptromega-p-tromega-p-qtromega-p}{%
\paragraph{\texorpdfstring{\[(\omega|_p \vDash q) = (\omega \vDash p \& q) / (\omega \vDash p) = Tr(\omega \sqrt p q \sqrt p)/Tr(\omega p)= Tr(\omega  p q)/Tr(\omega p) \]}{(\textbackslash{}omega\textbar{}\_p \textbackslash{}vDash q) = (\textbackslash{}omega \textbackslash{}vDash p \textbackslash{}\& q) / (\textbackslash{}omega \textbackslash{}vDash p) = Tr(\textbackslash{}omega \textbackslash{}sqrt p q \textbackslash{}sqrt p)/Tr(\textbackslash{}omega p)= Tr(\textbackslash{}omega  p q)/Tr(\textbackslash{}omega p) }}\label{omega_p-vdash-q-omega-vdash-p-q-omega-vdash-p-tromega-sqrt-p-q-sqrt-ptromega-p-tromega-p-qtromega-p}}

\[Tr(\omega p) = \sum_i \omega_i p_i = \]

\hypertarget{hence-if-omega-vdash-p-q-omega-vdash-q-p-i.e.}{%
\paragraph{\texorpdfstring{Hence if
\[ (\omega \vDash p \& q) = (\omega \vDash q \& p)\],
i.e.}{Hence if  (\textbackslash{}omega \textbackslash{}vDash p \textbackslash{}\& q) = (\textbackslash{}omega \textbackslash{}vDash q \textbackslash{}\& p), i.e.}}\label{hence-if-omega-vdash-p-q-omega-vdash-q-p-i.e.}}

\[Tr(\omega \sqrt{p}q\sqrt{p}) = Tr(\omega \sqrt{q}p\sqrt{q})\] then
@label-eq:condEff is @label-eq:bayes.

\hypertarget{diagonal-and-classical}{%
\subsection{Diagonal and classical}\label{diagonal-and-classical}}

\hypertarget{when-a-density-matrix-s-is-diagonal}{%
\subsubsection{\texorpdfstring{When a density matrix \[S\] is
diagonal}{When a density matrix S is diagonal}}\label{when-a-density-matrix-s-is-diagonal}}

\[S = diag(s_1,s_2,...,s_d)\]

\hypertarget{the-bayes-theorem-is-thus}{%
\subsubsection{The Bayes' theorem is
thus}\label{the-bayes-theorem-is-thus}}

\[Tr\] \#d-fill Because \#d-fill quantum prob is a prob different than
normal prob, such as Bayes prob.

\hypertarget{read}{%
\section{Read}\label{read}}

%\section{Bayes probability as $C^*$-algebra}

\begin{Def}A $C^*$-algebra $A = (|A|, \cdot, 1, \norm{-}, -^*)$ is a vector space over the complex numbers $\Bbb{C}$  such that for all $a,b \in |A|$, $\norm{a}^2 = a^* \cdot a$ and $\norm{a \cdot b} \leq \norm{a}\norm{b}$. A morphism $f : A \to A' $ between $C^*$-algebras is called as follows.

\noindent
\textit{Multiplicative (M)} when 

\begin{equation}
a\cdot b  \stackrel{f}{\longmapsto} f(a) \cdot' f(b). \label{eq:starM}
\end{equation}

\noindent
\textit{Unital (U)} when
\begin{equation}
 1 \stackrel{f}{\longmapsto} 1'. \label{eq:starU}
 \end{equation}

\noindent
\textit{Involutive (I)} when
\begin{equation}
f(a^*) = f(a)^*. \label{eq:starI}
\end{equation}

\noindent
\textit{Positive (P)} when
\begin{equation}
f(a) \geq  0'. \label{starP}
\end{equation}

\noindent
\textit{Completely positive (CP)} when
\begin{equation}
M_n(f):M_n(A) \to M_n(A') \label{eq:starCP}
\end{equation}
is positive for all $n \in \Bbb{N}$, where $M_n(A)$ is the $C^*$-algebra of $n$-by-$n$ matrices whose elements are in $A$, see Example \ref{eg:linalg} for details.
\end{Def}

\begin{Eg}\label{eg:Cs} The complex numbers $\Bbb{C}$. Let $|\Bbb{C}| = \{a+bi | a,b \in \Bbb{R} \} $, and for all $z,z' \in |\Bbb{C}|$,
\begin{alignat*}{2}
z^*         \coloneqq{}& \overline{z}, \\
\norm{z}    \coloneqq{}& \sqrt{z \overline{z}}, \\
z \cdot z'  \coloneqq{}& zz'.
\end{alignat*}
Obviously $\Cs = (|\Cs|, \cdot, 1, \norm{-}, -^*)$ forms a $C^*$-algebra.
\end{Eg}

\begin{Eg}\label{eg:linalg}
Matrix algebra of square matrices. $M_n(A) := (|M_n(A)|, X \cdot Y := XY, I_A, \norm{-}_{\infty}, (-)^\dagger)$
 for $n \in \Bbb{N}$ is a $C^*$-algebra. Notice
$M_n(\Cs) \cong \mathcal{B}_n := \mathcal{B}(\Cs^n) := \mathcal{B}(\Cs^n, \Cs^n)$.
\end{Eg}

Let \ctCstar{PU} be the category of $C^*$-algebras with positive unital maps between them as arrows, and \ctFH{isom} be the category of finite-dimensional Hilbert spaces and isometry maps. There are bijective correspondences about states and effects (predicates). For a predicate (effect) $p$,
\begin{equation} \label{bij:effect}
\begin{prooftree}
    \hypo{ \mathcal{B}_n &\xrightarrow{\mathmakebox[3em]{p}} 1+1 \text{   in \opposite{\ctCstar{PU}}}}
    \infer[double]1{ \mathcal{B}_n &\xleftarrow{\mathmakebox[3em]{p_{\varphi}}} \mathcal{B}_2 \text{   in \ctCstar{PU}}}
    \infer[double]1{\Cs^n &\xrightarrow{\mathmakebox[3em]{\varphi}} \Cs^2  \text{   in \ctFH{isom}}}
\end{prooftree}
\end{equation}
where $p_{\varphi}(A) := P^*AP$ with $P:=[\varphi(\ket{0}), ..., \varphi(\ket{n-1})]$, and $1$, resp. $\mathcal{B}_n$, is the $C^*$-algebra from Example \ref{eg:Cs}, resp. Example \ref{eg:linalg}.
For a state $\omega$,
\begin{equation} \label{bij:state}
\begin{prooftree}
    \hypo{ 1 \xrightarrow{\mathmakebox[3em]{\omega}} \mathcal{B}_n &\text{   in \opposite{\ctCstar{PU}}}}
    \infer[double]1{\Bbb{C} \xleftarrow{\mathmakebox[3em]{\omega_{\rho}}} \mathcal{B}_n &\text{   in \ctCstar{PU}}}
    \infer[double]1{\rho \in \DM{\Bbb{C}^n}}
\end{prooftree}
\end{equation}
where $\DM{\Cs^n} := \{\rho \in [0,1]_{\mathcal{B}_n}|\Tr{\rho}=1\}$, $[0,1]_A := \{a \in A| 0 \le a \le 1\}$, and  
\begin{equation} \label{eq:finstate}
\omega_{\rho}(p) = \Tr{\rho p}.    
\end{equation}

The Bayes' theorem
\[
Pr(q | p) = \frac{Pr(p \& q)}{Pr(p)}
\]
for $p,q \sim \omega$ is
\[
(\omega |_p \vDash q) := \frac{\omega \vDash p \& q}{\omega \vDash p}
\]
where 
\[
p \& q := \sqrt{p} \cdot q \cdot \sqrt{p},
\] \todo{What's the sqrt?}
and $\omega \vDash p$ is a compositon $ p \circ \omega$ in \opposite{\ctCstar{PU}}:
\[\begin{tikzcd}
1 \ar[r, "\omega"] \ar[bend right]{rr}[swap]{p \circ \omega} & \mathcal{H} \ar[r, "p"] & 1+1,
\end{tikzcd}\] that is, according to \eqref{bij:effect} and \eqref{bij:state}, $\omega_{\rho} \circ p_{\varphi}$ in \ctCstar{PU}: 
\[\begin{tikzcd}
\Cs  & \mathcal{H} \ar[swap, l, "\omega_{\rho}"]  & \Cs  \oplus \Cs, \ar[swap, l, "p_{\varphi}"] \ar[bend left]{ll}{\omega_{\rho} \circ p_{\varphi}}
\end{tikzcd}\] that is the Born rule: $\omega_{\rho}(p_{\varphi}) \stackrel{\eqref{eq:finstate}}{=} \Tr{\rho PP^*}$.

%\section{Progress and future works}

\printbibliography

%\appendix
\end{document}%%%%%%%%%%%%%%%%%%%%%%%%%%%%%%%%%%
